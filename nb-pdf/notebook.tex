
% Default to the notebook output style

    


% Inherit from the specified cell style.




    
\documentclass[11pt]{article}

    
    
    \usepackage[T1]{fontenc}
    % Nicer default font (+ math font) than Computer Modern for most use cases
    \usepackage{mathpazo}

    % Basic figure setup, for now with no caption control since it's done
    % automatically by Pandoc (which extracts ![](path) syntax from Markdown).
    \usepackage{graphicx}
    % We will generate all images so they have a width \maxwidth. This means
    % that they will get their normal width if they fit onto the page, but
    % are scaled down if they would overflow the margins.
    \makeatletter
    \def\maxwidth{\ifdim\Gin@nat@width>\linewidth\linewidth
    \else\Gin@nat@width\fi}
    \makeatother
    \let\Oldincludegraphics\includegraphics
    % Set max figure width to be 80% of text width, for now hardcoded.
    \renewcommand{\includegraphics}[1]{\Oldincludegraphics[width=.8\maxwidth]{#1}}
    % Ensure that by default, figures have no caption (until we provide a
    % proper Figure object with a Caption API and a way to capture that
    % in the conversion process - todo).
    \usepackage{caption}
    \DeclareCaptionLabelFormat{nolabel}{}
    \captionsetup{labelformat=nolabel}

    \usepackage{adjustbox} % Used to constrain images to a maximum size 
    \usepackage{xcolor} % Allow colors to be defined
    \usepackage{enumerate} % Needed for markdown enumerations to work
    \usepackage{geometry} % Used to adjust the document margins
    \usepackage{amsmath} % Equations
    \usepackage{amssymb} % Equations
    \usepackage{textcomp} % defines textquotesingle
    % Hack from http://tex.stackexchange.com/a/47451/13684:
    \AtBeginDocument{%
        \def\PYZsq{\textquotesingle}% Upright quotes in Pygmentized code
    }
    \usepackage{upquote} % Upright quotes for verbatim code
    \usepackage{eurosym} % defines \euro
    \usepackage[mathletters]{ucs} % Extended unicode (utf-8) support
    \usepackage[utf8x]{inputenc} % Allow utf-8 characters in the tex document
    \usepackage{fancyvrb} % verbatim replacement that allows latex
    \usepackage{grffile} % extends the file name processing of package graphics 
                         % to support a larger range 
    % The hyperref package gives us a pdf with properly built
    % internal navigation ('pdf bookmarks' for the table of contents,
    % internal cross-reference links, web links for URLs, etc.)
    \usepackage{hyperref}
    \usepackage{longtable} % longtable support required by pandoc >1.10
    \usepackage{booktabs}  % table support for pandoc > 1.12.2
    \usepackage[inline]{enumitem} % IRkernel/repr support (it uses the enumerate* environment)
    \usepackage[normalem]{ulem} % ulem is needed to support strikethroughs (\sout)
                                % normalem makes italics be italics, not underlines
    

    
    
    % Colors for the hyperref package
    \definecolor{urlcolor}{rgb}{0,.145,.698}
    \definecolor{linkcolor}{rgb}{.71,0.21,0.01}
    \definecolor{citecolor}{rgb}{.12,.54,.11}

    % ANSI colors
    \definecolor{ansi-black}{HTML}{3E424D}
    \definecolor{ansi-black-intense}{HTML}{282C36}
    \definecolor{ansi-red}{HTML}{E75C58}
    \definecolor{ansi-red-intense}{HTML}{B22B31}
    \definecolor{ansi-green}{HTML}{00A250}
    \definecolor{ansi-green-intense}{HTML}{007427}
    \definecolor{ansi-yellow}{HTML}{DDB62B}
    \definecolor{ansi-yellow-intense}{HTML}{B27D12}
    \definecolor{ansi-blue}{HTML}{208FFB}
    \definecolor{ansi-blue-intense}{HTML}{0065CA}
    \definecolor{ansi-magenta}{HTML}{D160C4}
    \definecolor{ansi-magenta-intense}{HTML}{A03196}
    \definecolor{ansi-cyan}{HTML}{60C6C8}
    \definecolor{ansi-cyan-intense}{HTML}{258F8F}
    \definecolor{ansi-white}{HTML}{C5C1B4}
    \definecolor{ansi-white-intense}{HTML}{A1A6B2}

    % commands and environments needed by pandoc snippets
    % extracted from the output of `pandoc -s`
    \providecommand{\tightlist}{%
      \setlength{\itemsep}{0pt}\setlength{\parskip}{0pt}}
    \DefineVerbatimEnvironment{Highlighting}{Verbatim}{commandchars=\\\{\}}
    % Add ',fontsize=\small' for more characters per line
    \newenvironment{Shaded}{}{}
    \newcommand{\KeywordTok}[1]{\textcolor[rgb]{0.00,0.44,0.13}{\textbf{{#1}}}}
    \newcommand{\DataTypeTok}[1]{\textcolor[rgb]{0.56,0.13,0.00}{{#1}}}
    \newcommand{\DecValTok}[1]{\textcolor[rgb]{0.25,0.63,0.44}{{#1}}}
    \newcommand{\BaseNTok}[1]{\textcolor[rgb]{0.25,0.63,0.44}{{#1}}}
    \newcommand{\FloatTok}[1]{\textcolor[rgb]{0.25,0.63,0.44}{{#1}}}
    \newcommand{\CharTok}[1]{\textcolor[rgb]{0.25,0.44,0.63}{{#1}}}
    \newcommand{\StringTok}[1]{\textcolor[rgb]{0.25,0.44,0.63}{{#1}}}
    \newcommand{\CommentTok}[1]{\textcolor[rgb]{0.38,0.63,0.69}{\textit{{#1}}}}
    \newcommand{\OtherTok}[1]{\textcolor[rgb]{0.00,0.44,0.13}{{#1}}}
    \newcommand{\AlertTok}[1]{\textcolor[rgb]{1.00,0.00,0.00}{\textbf{{#1}}}}
    \newcommand{\FunctionTok}[1]{\textcolor[rgb]{0.02,0.16,0.49}{{#1}}}
    \newcommand{\RegionMarkerTok}[1]{{#1}}
    \newcommand{\ErrorTok}[1]{\textcolor[rgb]{1.00,0.00,0.00}{\textbf{{#1}}}}
    \newcommand{\NormalTok}[1]{{#1}}
    
    % Additional commands for more recent versions of Pandoc
    \newcommand{\ConstantTok}[1]{\textcolor[rgb]{0.53,0.00,0.00}{{#1}}}
    \newcommand{\SpecialCharTok}[1]{\textcolor[rgb]{0.25,0.44,0.63}{{#1}}}
    \newcommand{\VerbatimStringTok}[1]{\textcolor[rgb]{0.25,0.44,0.63}{{#1}}}
    \newcommand{\SpecialStringTok}[1]{\textcolor[rgb]{0.73,0.40,0.53}{{#1}}}
    \newcommand{\ImportTok}[1]{{#1}}
    \newcommand{\DocumentationTok}[1]{\textcolor[rgb]{0.73,0.13,0.13}{\textit{{#1}}}}
    \newcommand{\AnnotationTok}[1]{\textcolor[rgb]{0.38,0.63,0.69}{\textbf{\textit{{#1}}}}}
    \newcommand{\CommentVarTok}[1]{\textcolor[rgb]{0.38,0.63,0.69}{\textbf{\textit{{#1}}}}}
    \newcommand{\VariableTok}[1]{\textcolor[rgb]{0.10,0.09,0.49}{{#1}}}
    \newcommand{\ControlFlowTok}[1]{\textcolor[rgb]{0.00,0.44,0.13}{\textbf{{#1}}}}
    \newcommand{\OperatorTok}[1]{\textcolor[rgb]{0.40,0.40,0.40}{{#1}}}
    \newcommand{\BuiltInTok}[1]{{#1}}
    \newcommand{\ExtensionTok}[1]{{#1}}
    \newcommand{\PreprocessorTok}[1]{\textcolor[rgb]{0.74,0.48,0.00}{{#1}}}
    \newcommand{\AttributeTok}[1]{\textcolor[rgb]{0.49,0.56,0.16}{{#1}}}
    \newcommand{\InformationTok}[1]{\textcolor[rgb]{0.38,0.63,0.69}{\textbf{\textit{{#1}}}}}
    \newcommand{\WarningTok}[1]{\textcolor[rgb]{0.38,0.63,0.69}{\textbf{\textit{{#1}}}}}
    
    
    % Define a nice break command that doesn't care if a line doesn't already
    % exist.
    \def\br{\hspace*{\fill} \\* }
    % Math Jax compatability definitions
    \def\gt{>}
    \def\lt{<}
    % Document parameters
    \title{1 - Regresion Lineal }
    
    
    

    % Pygments definitions
    
\makeatletter
\def\PY@reset{\let\PY@it=\relax \let\PY@bf=\relax%
    \let\PY@ul=\relax \let\PY@tc=\relax%
    \let\PY@bc=\relax \let\PY@ff=\relax}
\def\PY@tok#1{\csname PY@tok@#1\endcsname}
\def\PY@toks#1+{\ifx\relax#1\empty\else%
    \PY@tok{#1}\expandafter\PY@toks\fi}
\def\PY@do#1{\PY@bc{\PY@tc{\PY@ul{%
    \PY@it{\PY@bf{\PY@ff{#1}}}}}}}
\def\PY#1#2{\PY@reset\PY@toks#1+\relax+\PY@do{#2}}

\expandafter\def\csname PY@tok@w\endcsname{\def\PY@tc##1{\textcolor[rgb]{0.73,0.73,0.73}{##1}}}
\expandafter\def\csname PY@tok@c\endcsname{\let\PY@it=\textit\def\PY@tc##1{\textcolor[rgb]{0.25,0.50,0.50}{##1}}}
\expandafter\def\csname PY@tok@cp\endcsname{\def\PY@tc##1{\textcolor[rgb]{0.74,0.48,0.00}{##1}}}
\expandafter\def\csname PY@tok@k\endcsname{\let\PY@bf=\textbf\def\PY@tc##1{\textcolor[rgb]{0.00,0.50,0.00}{##1}}}
\expandafter\def\csname PY@tok@kp\endcsname{\def\PY@tc##1{\textcolor[rgb]{0.00,0.50,0.00}{##1}}}
\expandafter\def\csname PY@tok@kt\endcsname{\def\PY@tc##1{\textcolor[rgb]{0.69,0.00,0.25}{##1}}}
\expandafter\def\csname PY@tok@o\endcsname{\def\PY@tc##1{\textcolor[rgb]{0.40,0.40,0.40}{##1}}}
\expandafter\def\csname PY@tok@ow\endcsname{\let\PY@bf=\textbf\def\PY@tc##1{\textcolor[rgb]{0.67,0.13,1.00}{##1}}}
\expandafter\def\csname PY@tok@nb\endcsname{\def\PY@tc##1{\textcolor[rgb]{0.00,0.50,0.00}{##1}}}
\expandafter\def\csname PY@tok@nf\endcsname{\def\PY@tc##1{\textcolor[rgb]{0.00,0.00,1.00}{##1}}}
\expandafter\def\csname PY@tok@nc\endcsname{\let\PY@bf=\textbf\def\PY@tc##1{\textcolor[rgb]{0.00,0.00,1.00}{##1}}}
\expandafter\def\csname PY@tok@nn\endcsname{\let\PY@bf=\textbf\def\PY@tc##1{\textcolor[rgb]{0.00,0.00,1.00}{##1}}}
\expandafter\def\csname PY@tok@ne\endcsname{\let\PY@bf=\textbf\def\PY@tc##1{\textcolor[rgb]{0.82,0.25,0.23}{##1}}}
\expandafter\def\csname PY@tok@nv\endcsname{\def\PY@tc##1{\textcolor[rgb]{0.10,0.09,0.49}{##1}}}
\expandafter\def\csname PY@tok@no\endcsname{\def\PY@tc##1{\textcolor[rgb]{0.53,0.00,0.00}{##1}}}
\expandafter\def\csname PY@tok@nl\endcsname{\def\PY@tc##1{\textcolor[rgb]{0.63,0.63,0.00}{##1}}}
\expandafter\def\csname PY@tok@ni\endcsname{\let\PY@bf=\textbf\def\PY@tc##1{\textcolor[rgb]{0.60,0.60,0.60}{##1}}}
\expandafter\def\csname PY@tok@na\endcsname{\def\PY@tc##1{\textcolor[rgb]{0.49,0.56,0.16}{##1}}}
\expandafter\def\csname PY@tok@nt\endcsname{\let\PY@bf=\textbf\def\PY@tc##1{\textcolor[rgb]{0.00,0.50,0.00}{##1}}}
\expandafter\def\csname PY@tok@nd\endcsname{\def\PY@tc##1{\textcolor[rgb]{0.67,0.13,1.00}{##1}}}
\expandafter\def\csname PY@tok@s\endcsname{\def\PY@tc##1{\textcolor[rgb]{0.73,0.13,0.13}{##1}}}
\expandafter\def\csname PY@tok@sd\endcsname{\let\PY@it=\textit\def\PY@tc##1{\textcolor[rgb]{0.73,0.13,0.13}{##1}}}
\expandafter\def\csname PY@tok@si\endcsname{\let\PY@bf=\textbf\def\PY@tc##1{\textcolor[rgb]{0.73,0.40,0.53}{##1}}}
\expandafter\def\csname PY@tok@se\endcsname{\let\PY@bf=\textbf\def\PY@tc##1{\textcolor[rgb]{0.73,0.40,0.13}{##1}}}
\expandafter\def\csname PY@tok@sr\endcsname{\def\PY@tc##1{\textcolor[rgb]{0.73,0.40,0.53}{##1}}}
\expandafter\def\csname PY@tok@ss\endcsname{\def\PY@tc##1{\textcolor[rgb]{0.10,0.09,0.49}{##1}}}
\expandafter\def\csname PY@tok@sx\endcsname{\def\PY@tc##1{\textcolor[rgb]{0.00,0.50,0.00}{##1}}}
\expandafter\def\csname PY@tok@m\endcsname{\def\PY@tc##1{\textcolor[rgb]{0.40,0.40,0.40}{##1}}}
\expandafter\def\csname PY@tok@gh\endcsname{\let\PY@bf=\textbf\def\PY@tc##1{\textcolor[rgb]{0.00,0.00,0.50}{##1}}}
\expandafter\def\csname PY@tok@gu\endcsname{\let\PY@bf=\textbf\def\PY@tc##1{\textcolor[rgb]{0.50,0.00,0.50}{##1}}}
\expandafter\def\csname PY@tok@gd\endcsname{\def\PY@tc##1{\textcolor[rgb]{0.63,0.00,0.00}{##1}}}
\expandafter\def\csname PY@tok@gi\endcsname{\def\PY@tc##1{\textcolor[rgb]{0.00,0.63,0.00}{##1}}}
\expandafter\def\csname PY@tok@gr\endcsname{\def\PY@tc##1{\textcolor[rgb]{1.00,0.00,0.00}{##1}}}
\expandafter\def\csname PY@tok@ge\endcsname{\let\PY@it=\textit}
\expandafter\def\csname PY@tok@gs\endcsname{\let\PY@bf=\textbf}
\expandafter\def\csname PY@tok@gp\endcsname{\let\PY@bf=\textbf\def\PY@tc##1{\textcolor[rgb]{0.00,0.00,0.50}{##1}}}
\expandafter\def\csname PY@tok@go\endcsname{\def\PY@tc##1{\textcolor[rgb]{0.53,0.53,0.53}{##1}}}
\expandafter\def\csname PY@tok@gt\endcsname{\def\PY@tc##1{\textcolor[rgb]{0.00,0.27,0.87}{##1}}}
\expandafter\def\csname PY@tok@err\endcsname{\def\PY@bc##1{\setlength{\fboxsep}{0pt}\fcolorbox[rgb]{1.00,0.00,0.00}{1,1,1}{\strut ##1}}}
\expandafter\def\csname PY@tok@kc\endcsname{\let\PY@bf=\textbf\def\PY@tc##1{\textcolor[rgb]{0.00,0.50,0.00}{##1}}}
\expandafter\def\csname PY@tok@kd\endcsname{\let\PY@bf=\textbf\def\PY@tc##1{\textcolor[rgb]{0.00,0.50,0.00}{##1}}}
\expandafter\def\csname PY@tok@kn\endcsname{\let\PY@bf=\textbf\def\PY@tc##1{\textcolor[rgb]{0.00,0.50,0.00}{##1}}}
\expandafter\def\csname PY@tok@kr\endcsname{\let\PY@bf=\textbf\def\PY@tc##1{\textcolor[rgb]{0.00,0.50,0.00}{##1}}}
\expandafter\def\csname PY@tok@bp\endcsname{\def\PY@tc##1{\textcolor[rgb]{0.00,0.50,0.00}{##1}}}
\expandafter\def\csname PY@tok@fm\endcsname{\def\PY@tc##1{\textcolor[rgb]{0.00,0.00,1.00}{##1}}}
\expandafter\def\csname PY@tok@vc\endcsname{\def\PY@tc##1{\textcolor[rgb]{0.10,0.09,0.49}{##1}}}
\expandafter\def\csname PY@tok@vg\endcsname{\def\PY@tc##1{\textcolor[rgb]{0.10,0.09,0.49}{##1}}}
\expandafter\def\csname PY@tok@vi\endcsname{\def\PY@tc##1{\textcolor[rgb]{0.10,0.09,0.49}{##1}}}
\expandafter\def\csname PY@tok@vm\endcsname{\def\PY@tc##1{\textcolor[rgb]{0.10,0.09,0.49}{##1}}}
\expandafter\def\csname PY@tok@sa\endcsname{\def\PY@tc##1{\textcolor[rgb]{0.73,0.13,0.13}{##1}}}
\expandafter\def\csname PY@tok@sb\endcsname{\def\PY@tc##1{\textcolor[rgb]{0.73,0.13,0.13}{##1}}}
\expandafter\def\csname PY@tok@sc\endcsname{\def\PY@tc##1{\textcolor[rgb]{0.73,0.13,0.13}{##1}}}
\expandafter\def\csname PY@tok@dl\endcsname{\def\PY@tc##1{\textcolor[rgb]{0.73,0.13,0.13}{##1}}}
\expandafter\def\csname PY@tok@s2\endcsname{\def\PY@tc##1{\textcolor[rgb]{0.73,0.13,0.13}{##1}}}
\expandafter\def\csname PY@tok@sh\endcsname{\def\PY@tc##1{\textcolor[rgb]{0.73,0.13,0.13}{##1}}}
\expandafter\def\csname PY@tok@s1\endcsname{\def\PY@tc##1{\textcolor[rgb]{0.73,0.13,0.13}{##1}}}
\expandafter\def\csname PY@tok@mb\endcsname{\def\PY@tc##1{\textcolor[rgb]{0.40,0.40,0.40}{##1}}}
\expandafter\def\csname PY@tok@mf\endcsname{\def\PY@tc##1{\textcolor[rgb]{0.40,0.40,0.40}{##1}}}
\expandafter\def\csname PY@tok@mh\endcsname{\def\PY@tc##1{\textcolor[rgb]{0.40,0.40,0.40}{##1}}}
\expandafter\def\csname PY@tok@mi\endcsname{\def\PY@tc##1{\textcolor[rgb]{0.40,0.40,0.40}{##1}}}
\expandafter\def\csname PY@tok@il\endcsname{\def\PY@tc##1{\textcolor[rgb]{0.40,0.40,0.40}{##1}}}
\expandafter\def\csname PY@tok@mo\endcsname{\def\PY@tc##1{\textcolor[rgb]{0.40,0.40,0.40}{##1}}}
\expandafter\def\csname PY@tok@ch\endcsname{\let\PY@it=\textit\def\PY@tc##1{\textcolor[rgb]{0.25,0.50,0.50}{##1}}}
\expandafter\def\csname PY@tok@cm\endcsname{\let\PY@it=\textit\def\PY@tc##1{\textcolor[rgb]{0.25,0.50,0.50}{##1}}}
\expandafter\def\csname PY@tok@cpf\endcsname{\let\PY@it=\textit\def\PY@tc##1{\textcolor[rgb]{0.25,0.50,0.50}{##1}}}
\expandafter\def\csname PY@tok@c1\endcsname{\let\PY@it=\textit\def\PY@tc##1{\textcolor[rgb]{0.25,0.50,0.50}{##1}}}
\expandafter\def\csname PY@tok@cs\endcsname{\let\PY@it=\textit\def\PY@tc##1{\textcolor[rgb]{0.25,0.50,0.50}{##1}}}

\def\PYZbs{\char`\\}
\def\PYZus{\char`\_}
\def\PYZob{\char`\{}
\def\PYZcb{\char`\}}
\def\PYZca{\char`\^}
\def\PYZam{\char`\&}
\def\PYZlt{\char`\<}
\def\PYZgt{\char`\>}
\def\PYZsh{\char`\#}
\def\PYZpc{\char`\%}
\def\PYZdl{\char`\$}
\def\PYZhy{\char`\-}
\def\PYZsq{\char`\'}
\def\PYZdq{\char`\"}
\def\PYZti{\char`\~}
% for compatibility with earlier versions
\def\PYZat{@}
\def\PYZlb{[}
\def\PYZrb{]}
\makeatother


    % Exact colors from NB
    \definecolor{incolor}{rgb}{0.0, 0.0, 0.5}
    \definecolor{outcolor}{rgb}{0.545, 0.0, 0.0}



    
    % Prevent overflowing lines due to hard-to-break entities
    \sloppy 
    % Setup hyperref package
    \hypersetup{
      breaklinks=true,  % so long urls are correctly broken across lines
      colorlinks=true,
      urlcolor=urlcolor,
      linkcolor=linkcolor,
      citecolor=citecolor,
      }
    % Slightly bigger margins than the latex defaults
    
    \geometry{verbose,tmargin=1in,bmargin=1in,lmargin=1in,rmargin=1in}
    
    

    \begin{document}
    
    
    \maketitle
    
    

    
    \section{Parte 1 : Regresión lineal con una
variable}\label{parte-1-regresiuxf3n-lineal-con-una-variable}

    \emph{\textbf{Autores: Alberto Pastor Moreno e Iván Fernández Mena}}

    \subsection{Carga de datos}\label{carga-de-datos}

    En la práctica uno se va a trabajar la regresion lineal con una variable
a partir de los datos almacenados en un archivo csv que se nos
proporciona. Los datos extraidos estan separado en dos columnas y
representan los beneficios de una compañia de distribución de comida en
distintas ciudades en base a su población.

    A continuación se importaran todas las librerias necesarias para hacer
esta practica, tanto para leer el csv como numpy para el soporte de
vectores y matrices.

    \begin{Verbatim}[commandchars=\\\{\}]
{\color{incolor}In [{\color{incolor}1}]:} \PY{k+kn}{from} \PY{n+nn}{pandas}\PY{n+nn}{.}\PY{n+nn}{io}\PY{n+nn}{.}\PY{n+nn}{parsers} \PY{k}{import} \PY{n}{read\PYZus{}csv}
        \PY{k+kn}{import} \PY{n+nn}{numpy} \PY{k}{as} \PY{n+nn}{np}
\end{Verbatim}


    Definimos un método que lee y carga los datos de un csv y retorna esta
información recopilada. Se dicta que en el archivo no hay información de
cabecera (header = None) y que los valores que son devueltos son de tipo
flotantes.

    \begin{Verbatim}[commandchars=\\\{\}]
{\color{incolor}In [{\color{incolor}2}]:} \PY{k}{def} \PY{n+nf}{load\PYZus{}csv}\PY{p}{(}\PY{n}{filename}\PY{p}{)}\PY{p}{:}
            \PY{n}{values} \PY{o}{=} \PY{n}{read\PYZus{}csv}\PY{p}{(}\PY{n}{filename}\PY{p}{,} \PY{n}{header}\PY{o}{=}\PY{k+kc}{None}\PY{p}{)}\PY{o}{.}\PY{n}{values}
            \PY{k}{return} \PY{n}{values}\PY{o}{.}\PY{n}{astype}\PY{p}{(}\PY{n+nb}{float}\PY{p}{)}
\end{Verbatim}


    \begin{Verbatim}[commandchars=\\\{\}]
{\color{incolor}In [{\color{incolor}3}]:} \PY{n}{dataset} \PY{o}{=} \PY{n}{load\PYZus{}csv}\PY{p}{(}\PY{l+s+s1}{\PYZsq{}}\PY{l+s+s1}{./datasets/ex1data1.csv}\PY{l+s+s1}{\PYZsq{}}\PY{p}{)}
\end{Verbatim}


    Utilizamos el conjunto de datos obtenido del csv con la función
especificada previamente y separamos cada columna en vectores diferentes
para poder gestionarlos de manera independiente, de este modo de puede
usar cada columna para su estudio.

    \begin{Verbatim}[commandchars=\\\{\}]
{\color{incolor}In [{\color{incolor}4}]:} \PY{n}{independent\PYZus{}data} \PY{o}{=} \PY{n}{dataset}\PY{p}{[}\PY{p}{:}\PY{p}{,}\PY{l+m+mi}{0}\PY{p}{]}
        \PY{n}{dependent\PYZus{}data} \PY{o}{=} \PY{n}{dataset}\PY{p}{[}\PY{p}{:}\PY{p}{,}\PY{l+m+mi}{1}\PY{p}{]}
\end{Verbatim}


    \subsection{Visualización simple de
dataset}\label{visualizaciuxf3n-simple-de-dataset}

    A continuación se generará un plotter con los datos que previamente
tratados. Se creará un plotter de cruces enfrentando los valores de
beneficios de la empresa estudiada y la población que se nos ha
proporcionado en el dataset.

    \begin{Verbatim}[commandchars=\\\{\}]
{\color{incolor}In [{\color{incolor}5}]:} \PY{k+kn}{import} \PY{n+nn}{matplotlib}\PY{n+nn}{.}\PY{n+nn}{pyplot} \PY{k}{as} \PY{n+nn}{plt}
\end{Verbatim}


    \begin{Verbatim}[commandchars=\\\{\}]
{\color{incolor}In [{\color{incolor}6}]:} \PY{n}{plt}\PY{o}{.}\PY{n}{figure}\PY{p}{(}\PY{p}{)}
        \PY{n}{plt}\PY{o}{.}\PY{n}{plot}\PY{p}{(}\PY{n}{independent\PYZus{}data}\PY{p}{,} \PY{n}{dependent\PYZus{}data}\PY{p}{,} \PY{l+s+s1}{\PYZsq{}}\PY{l+s+s1}{rx}\PY{l+s+s1}{\PYZsq{}}\PY{p}{)}
        \PY{n}{plt}\PY{o}{.}\PY{n}{ylabel}\PY{p}{(}\PY{l+s+s1}{\PYZsq{}}\PY{l+s+s1}{Ingresos en \PYZdl{}10.000s}\PY{l+s+s1}{\PYZsq{}}\PY{p}{)}
        \PY{n}{plt}\PY{o}{.}\PY{n}{xlabel}\PY{p}{(}\PY{l+s+s1}{\PYZsq{}}\PY{l+s+s1}{Poblacion de la ciudad en 10.000s}\PY{l+s+s1}{\PYZsq{}}\PY{p}{)}
\end{Verbatim}


\begin{Verbatim}[commandchars=\\\{\}]
{\color{outcolor}Out[{\color{outcolor}6}]:} Text(0.5,0,'Poblacion de la ciudad en 10.000s')
\end{Verbatim}
            
    \begin{center}
    \adjustimage{max size={0.9\linewidth}{0.9\paperheight}}{output_14_1.png}
    \end{center}
    { \hspace*{\fill} \\}
    
    Como se observa en el plotter generado, ya es posible hacer un estudio
previo sobre los datos, llegando a la conclusión inicial en que en la
mayoría de los datos, según aumente el número de ingresos también
aumenta la población de la ciudad. Se puede hacer la hipótesis
asimilando que el modelo válido sea la regresión lineal.

    \subsection{Método de descenso
gradiente}\label{muxe9todo-de-descenso-gradiente}

    Para encontrar la regresión lineal que más se adapte a nuestro dataset,
es necesario realizar operaciones básicas de modo que en cada una de las
iteraciones se obtenga un resultado más cercano al óptimo. Se necesita
definir la función de coste y la función de gradiente de descenso que
nos permitira obtener este resultado.

Siempre se tiene que tener en cuenta la función que se usará como
hipótesis, en nuestro caso se trata de una unica variable y de una
recta.

    \subsubsection{Implementación de la función de hipótesis de la regresión
lineal}\label{implementaciuxf3n-de-la-funciuxf3n-de-hipuxf3tesis-de-la-regresiuxf3n-lineal}

    \begin{Verbatim}[commandchars=\\\{\}]
{\color{incolor}In [{\color{incolor}7}]:} \PY{k}{def} \PY{n+nf}{hyphotesis\PYZus{}function}\PY{p}{(}\PY{n}{th0}\PY{p}{,} \PY{n}{th1}\PY{p}{,} \PY{n}{x}\PY{p}{)}\PY{p}{:}
            \PY{k}{return} \PY{n}{th0} \PY{o}{+} \PY{n}{th1}\PY{o}{*}\PY{n}{x}
\end{Verbatim}


    \subsubsection{\texorpdfstring{Implementacion de la función de coste de
la regresion lineal
J(\(\theta_0\),\(\theta_1\))}{Implementacion de la función de coste de la regresion lineal J(\textbackslash{}theta\_0,\textbackslash{}theta\_1)}}\label{implementacion-de-la-funciuxf3n-de-coste-de-la-regresion-lineal-jtheta_0theta_1}

    \begin{Verbatim}[commandchars=\\\{\}]
{\color{incolor}In [{\color{incolor}8}]:} \PY{k}{def} \PY{n+nf}{cost\PYZus{}function}\PY{p}{(}\PY{n}{fun}\PY{p}{,} \PY{n}{th0}\PY{p}{,} \PY{n}{th1}\PY{p}{,} \PY{n}{m}\PY{p}{,} \PY{n}{x}\PY{p}{,} \PY{n}{y}\PY{p}{)}\PY{p}{:}
            \PY{n}{sum\PYZus{}cost} \PY{o}{=} \PY{l+m+mi}{0}
            \PY{k}{for} \PY{n}{i} \PY{o+ow}{in} \PY{n+nb}{range}\PY{p}{(}\PY{l+m+mi}{0}\PY{p}{,} \PY{n}{m}\PY{p}{)}\PY{p}{:}
                \PY{n}{sum\PYZus{}cost} \PY{o}{+}\PY{o}{=} \PY{p}{(}\PY{n}{fun}\PY{p}{(}\PY{n}{th0}\PY{p}{,} \PY{n}{th1}\PY{p}{,} \PY{n}{x}\PY{p}{[}\PY{n}{i}\PY{p}{]}\PY{p}{)} \PY{o}{\PYZhy{}} \PY{n}{y}\PY{p}{[}\PY{n}{i}\PY{p}{]}\PY{p}{)}\PY{o}{*}\PY{o}{*}\PY{l+m+mi}{2}
            \PY{n}{cost} \PY{o}{=} \PY{n}{sum\PYZus{}cost} \PY{o}{/} \PY{p}{(}\PY{l+m+mi}{2}\PY{o}{*}\PY{n}{m}\PY{p}{)}
            \PY{k}{return} \PY{n}{cost}
\end{Verbatim}


    \subsubsection{\texorpdfstring{Implementación de función \emph{gradient
descent}}{Implementación de función gradient descent}}\label{implementaciuxf3n-de-funciuxf3n-gradient-descent}

    \begin{Verbatim}[commandchars=\\\{\}]
{\color{incolor}In [{\color{incolor}10}]:} \PY{k}{def} \PY{n+nf}{gradient\PYZus{}descent}\PY{p}{(}\PY{n}{fun}\PY{p}{,} \PY{n}{th0}\PY{p}{,} \PY{n}{th1}\PY{p}{,} \PY{n}{m}\PY{p}{,} \PY{n}{x}\PY{p}{,} \PY{n}{y}\PY{p}{,} \PY{n}{lr}\PY{o}{=}\PY{l+m+mf}{0.01}\PY{p}{,} \PY{n}{epochs}\PY{o}{=}\PY{l+m+mi}{1500}\PY{p}{)}\PY{p}{:}
             \PY{n}{cost} \PY{o}{=} \PY{p}{[}\PY{p}{]}
             \PY{n}{vc\PYZus{}th0} \PY{o}{=} \PY{p}{[}\PY{p}{]}
             \PY{n}{vc\PYZus{}th1} \PY{o}{=} \PY{p}{[}\PY{p}{]}
             \PY{n}{curr\PYZus{}th0} \PY{o}{=} \PY{n}{th0}
             \PY{n}{curr\PYZus{}th1} \PY{o}{=} \PY{n}{th1}
             \PY{k}{for} \PY{n}{i} \PY{o+ow}{in} \PY{n+nb}{range}\PY{p}{(}\PY{l+m+mi}{0}\PY{p}{,} \PY{n}{epochs}\PY{p}{)}\PY{p}{:}
                 \PY{n}{new\PYZus{}th0} \PY{o}{=} \PY{n}{curr\PYZus{}th0} \PY{o}{\PYZhy{}} \PY{p}{(}\PY{n}{lr}\PY{o}{/}\PY{n}{m}\PY{p}{)}\PY{o}{*}\PY{n}{np}\PY{o}{.}\PY{n}{sum}\PY{p}{(}\PY{p}{[}\PY{n}{fun}\PY{p}{(}\PY{n}{curr\PYZus{}th0}\PY{p}{,} \PY{n}{curr\PYZus{}th1}\PY{p}{,} \PY{n}{x}\PY{p}{[}\PY{n}{j}\PY{p}{]}\PY{p}{)} \PY{o}{\PYZhy{}} \PY{n}{y}\PY{p}{[}\PY{n}{j}\PY{p}{]}
                                                     \PY{k}{for} \PY{n}{j} \PY{o+ow}{in} \PY{n+nb}{range}\PY{p}{(}\PY{l+m+mi}{0}\PY{p}{,} \PY{n}{m}\PY{p}{)}\PY{p}{]}\PY{p}{)}
                 \PY{n}{new\PYZus{}th1} \PY{o}{=} \PY{n}{curr\PYZus{}th1} \PY{o}{\PYZhy{}} \PY{p}{(}\PY{n}{lr}\PY{o}{/}\PY{n}{m}\PY{p}{)}\PY{o}{*}\PY{n}{np}\PY{o}{.}\PY{n}{sum}\PY{p}{(}\PY{p}{[}\PY{p}{(}\PY{p}{(}\PY{n}{fun}\PY{p}{(}\PY{n}{curr\PYZus{}th0}\PY{p}{,} \PY{n}{curr\PYZus{}th1}\PY{p}{,} \PY{n}{x}\PY{p}{[}\PY{n}{j}\PY{p}{]}\PY{p}{)} \PY{o}{\PYZhy{}} \PY{n}{y}\PY{p}{[}\PY{n}{j}\PY{p}{]}\PY{p}{)}\PY{o}{*}\PY{n}{x}\PY{p}{[}\PY{n}{j}\PY{p}{]}\PY{p}{)}
                                                     \PY{k}{for} \PY{n}{j} \PY{o+ow}{in} \PY{n+nb}{range}\PY{p}{(}\PY{l+m+mi}{0}\PY{p}{,} \PY{n}{m}\PY{p}{)}\PY{p}{]}\PY{p}{)}
                 \PY{n}{curr\PYZus{}th0} \PY{o}{=} \PY{n}{new\PYZus{}th0}
                 \PY{n}{curr\PYZus{}th1} \PY{o}{=} \PY{n}{new\PYZus{}th1}
                 \PY{n}{epoch\PYZus{}cost} \PY{o}{=} \PY{n}{cost\PYZus{}function}\PY{p}{(}\PY{n}{fun}\PY{p}{,} \PY{n}{curr\PYZus{}th0}\PY{p}{,} \PY{n}{curr\PYZus{}th1}\PY{p}{,} \PY{n}{m}\PY{p}{,} \PY{n}{x}\PY{p}{,} \PY{n}{y}\PY{p}{)}
                 \PY{n}{cost} \PY{o}{+}\PY{o}{=} \PY{p}{[}\PY{n}{epoch\PYZus{}cost}\PY{p}{]}
                 \PY{n}{vc\PYZus{}th0} \PY{o}{+}\PY{o}{=} \PY{p}{[}\PY{n}{curr\PYZus{}th0}\PY{p}{]}
                 \PY{n}{vc\PYZus{}th1} \PY{o}{+}\PY{o}{=} \PY{p}{[}\PY{n}{curr\PYZus{}th1}\PY{p}{]}
             \PY{k}{return} \PY{n}{curr\PYZus{}th0}\PY{p}{,} \PY{n}{curr\PYZus{}th1}\PY{p}{,} \PY{n}{cost}\PY{p}{,} \PY{n}{vc\PYZus{}th0}\PY{p}{,} \PY{n}{vc\PYZus{}th1} 
\end{Verbatim}


    A continución se muestran las variables que se han usado para la gestión
de la práctica, de este modo se pueden cambiar paramentros de las
pruebas de forma comoda.

    \begin{Verbatim}[commandchars=\\\{\}]
{\color{incolor}In [{\color{incolor}11}]:} \PY{n}{th0} \PY{o}{=} \PY{l+m+mi}{0}
         \PY{n}{th1} \PY{o}{=} \PY{l+m+mi}{0}
         \PY{n}{lr} \PY{o}{=} \PY{l+m+mf}{0.01}
         \PY{n}{m} \PY{o}{=} \PY{n+nb}{len}\PY{p}{(}\PY{n}{dataset}\PY{p}{)}
\end{Verbatim}


    \subsection{Resultados obtenidos del
estudio}\label{resultados-obtenidos-del-estudio}

    Se aplican las funciones definidas previamente y se muestran los
resultados para poder tomar unas conclusiones concretas. Se ejecuta la
función de descenso del gradiente en nuestra función de hipótesis,
además de toda la información necesaria gestionada y obtenida de nuestro
conjunto de datos.

    \begin{Verbatim}[commandchars=\\\{\}]
{\color{incolor}In [{\color{incolor}12}]:} \PY{n}{gd\PYZus{}th0}\PY{p}{,} \PY{n}{gd\PYZus{}th1}\PY{p}{,} \PY{n}{gd\PYZus{}cost}\PY{p}{,} \PY{n}{vc\PYZus{}gd\PYZus{}th0}\PY{p}{,} \PY{n}{vc\PYZus{}gd\PYZus{}th1} \PY{o}{=} \PY{n}{gradient\PYZus{}descent}\PY{p}{(}\PY{n}{hyphotesis\PYZus{}function}\PY{p}{,}
                                                                          \PY{n}{th0}\PY{p}{,} \PY{n}{th1}\PY{p}{,} \PY{n}{m}\PY{p}{,} \PY{n}{independent\PYZus{}data}\PY{p}{,}
                                                                          \PY{n}{dependent\PYZus{}data}\PY{p}{,} \PY{n}{lr}\PY{p}{)}
         \PY{n+nb}{print} \PY{p}{(}\PY{l+s+s1}{\PYZsq{}}\PY{l+s+s1}{th0:}\PY{l+s+si}{\PYZob{}\PYZcb{}}\PY{l+s+s1}{, th1:}\PY{l+s+si}{\PYZob{}\PYZcb{}}\PY{l+s+s1}{, cost:}\PY{l+s+si}{\PYZob{}\PYZcb{}}\PY{l+s+s1}{\PYZsq{}}\PY{o}{.}\PY{n}{format}\PY{p}{(}\PY{n}{gd\PYZus{}th0}\PY{p}{,} \PY{n}{gd\PYZus{}th1}\PY{p}{,} \PY{n}{gd\PYZus{}cost}\PY{p}{[}\PY{o}{\PYZhy{}}\PY{l+m+mi}{1}\PY{p}{]}\PY{p}{)}\PY{p}{)}
\end{Verbatim}


    \begin{Verbatim}[commandchars=\\\{\}]
th0:-3.6302914394043606, th1:1.166362350335582, cost:4.483388256587728

    \end{Verbatim}

    Como se estudia en los datos mostrados previamente y en la gráfica que
se genera posteriormente, se observa que nuestro valor de coste va
disminuyendo según se aplica el descenso de gradiente. Es importante que
este coste baje de manera constante, una cantidad mayor al principio y
después se estabilice. Si esta acción no se esta realizando es síntoma
de que alguna acción esta mal.

    \begin{Verbatim}[commandchars=\\\{\}]
{\color{incolor}In [{\color{incolor}13}]:} \PY{n}{plt}\PY{o}{.}\PY{n}{plot}\PY{p}{(}\PY{n}{gd\PYZus{}cost}\PY{p}{)}
         \PY{n}{plt}\PY{o}{.}\PY{n}{title}\PY{p}{(}\PY{l+s+s1}{\PYZsq{}}\PY{l+s+s1}{Coste por iteración en el método gradient\PYZus{}descent()}\PY{l+s+s1}{\PYZsq{}}\PY{p}{)}
         \PY{n}{plt}\PY{o}{.}\PY{n}{ylabel}\PY{p}{(}\PY{l+s+s1}{\PYZsq{}}\PY{l+s+s1}{Cost}\PY{l+s+s1}{\PYZsq{}}\PY{p}{)}
         \PY{n}{plt}\PY{o}{.}\PY{n}{xlabel}\PY{p}{(}\PY{l+s+s1}{\PYZsq{}}\PY{l+s+s1}{num\PYZus{}epoch}\PY{l+s+s1}{\PYZsq{}}\PY{p}{)}
\end{Verbatim}


\begin{Verbatim}[commandchars=\\\{\}]
{\color{outcolor}Out[{\color{outcolor}13}]:} Text(0.5,0,'num\_epoch')
\end{Verbatim}
            
    \begin{center}
    \adjustimage{max size={0.9\linewidth}{0.9\paperheight}}{output_30_1.png}
    \end{center}
    { \hspace*{\fill} \\}
    
    \subsubsection{Visualización en plotter
normal}\label{visualizaciuxf3n-en-plotter-normal}

    Finalmente se muestra el resultado de manera coherente enfrentando los
valores del dataset original y los datos obtenidos de ejecutar nuestra
función gradiente sobre esos datos.

    \begin{Verbatim}[commandchars=\\\{\}]
{\color{incolor}In [{\color{incolor}14}]:} \PY{n}{plt}\PY{o}{.}\PY{n}{plot}\PY{p}{(}\PY{n}{independent\PYZus{}data}\PY{p}{,} \PY{n}{dependent\PYZus{}data}\PY{p}{,} \PY{l+s+s1}{\PYZsq{}}\PY{l+s+s1}{rx}\PY{l+s+s1}{\PYZsq{}}\PY{p}{,}
                  \PY{n}{independent\PYZus{}data}\PY{p}{,} \PY{n}{hyphotesis\PYZus{}function}\PY{p}{(}\PY{n}{gd\PYZus{}th0}\PY{p}{,} \PY{n}{gd\PYZus{}th1}\PY{p}{,} \PY{n}{independent\PYZus{}data}\PY{p}{)}\PY{p}{)}
         \PY{n}{plt}\PY{o}{.}\PY{n}{title}\PY{p}{(}\PY{l+s+s1}{\PYZsq{}}\PY{l+s+s1}{Recta resultado gradient\PYZus{}descent()}\PY{l+s+s1}{\PYZsq{}}\PY{p}{)}
         \PY{n}{plt}\PY{o}{.}\PY{n}{ylabel}\PY{p}{(}\PY{l+s+s1}{\PYZsq{}}\PY{l+s+s1}{Ingresos en \PYZdl{}10.000s}\PY{l+s+s1}{\PYZsq{}}\PY{p}{)}
         \PY{n}{plt}\PY{o}{.}\PY{n}{xlabel}\PY{p}{(}\PY{l+s+s1}{\PYZsq{}}\PY{l+s+s1}{Poblacion de la ciudad en 10.000s}\PY{l+s+s1}{\PYZsq{}}\PY{p}{)}
\end{Verbatim}


\begin{Verbatim}[commandchars=\\\{\}]
{\color{outcolor}Out[{\color{outcolor}14}]:} Text(0.5,0,'Poblacion de la ciudad en 10.000s')
\end{Verbatim}
            
    \begin{center}
    \adjustimage{max size={0.9\linewidth}{0.9\paperheight}}{output_33_1.png}
    \end{center}
    { \hspace*{\fill} \\}
    
    El resultado es una recta que se encuentra ajustada lo máximo posible
para nuestro conjunto de datos, esta recta tiene la separación mínima
entre los puntos y la resta, obtenido gracias a la función descenso de
gradiente.

    \subsubsection{Visualización en plotters
3D}\label{visualizaciuxf3n-en-plotters-3d}

    A continuación se monstrará el resultado obtenido en gráficas con mas de
dos dimensiones. Estas gráficas son importantes para el análisis del
descenso del gradiente de forma visual y para el estudio de aprendizajes
mas complejos.

    \begin{Verbatim}[commandchars=\\\{\}]
{\color{incolor}In [{\color{incolor}19}]:} \PY{k+kn}{from} \PY{n+nn}{mpl\PYZus{}toolkits}\PY{n+nn}{.}\PY{n+nn}{mplot3d} \PY{k}{import} \PY{n}{Axes3D}
         \PY{k+kn}{from} \PY{n+nn}{matplotlib}\PY{n+nn}{.}\PY{n+nn}{pyplot} \PY{k}{import} \PY{n}{cm}
         \PY{k+kn}{from} \PY{n+nn}{matplotlib}\PY{n+nn}{.}\PY{n+nn}{ticker} \PY{k}{import} \PY{n}{LinearLocator}\PY{p}{,} \PY{n}{FormatStrFormatter}
\end{Verbatim}


    \begin{Verbatim}[commandchars=\\\{\}]
{\color{incolor}In [{\color{incolor}20}]:} \PY{n}{fig} \PY{o}{=} \PY{n}{plt}\PY{o}{.}\PY{n}{figure}\PY{p}{(}\PY{p}{)}
         \PY{n}{ax} \PY{o}{=} \PY{n}{Axes3D}\PY{p}{(}\PY{n}{fig}\PY{p}{)}
         
         \PY{n}{X}\PY{p}{,} \PY{n}{Y} \PY{o}{=} \PY{n}{np}\PY{o}{.}\PY{n}{meshgrid}\PY{p}{(}\PY{n}{np}\PY{o}{.}\PY{n}{arange}\PY{p}{(}\PY{o}{\PYZhy{}}\PY{l+m+mi}{10}\PY{p}{,} \PY{l+m+mi}{10}\PY{p}{,} \PY{l+m+mf}{0.1}\PY{p}{)}\PY{p}{,} \PY{n}{np}\PY{o}{.}\PY{n}{arange}\PY{p}{(}\PY{o}{\PYZhy{}}\PY{l+m+mi}{1}\PY{p}{,} \PY{l+m+mi}{4}\PY{p}{,} \PY{l+m+mf}{0.1}\PY{p}{)}\PY{p}{)}
         \PY{n}{Z} \PY{o}{=} \PY{n}{cost\PYZus{}function}\PY{p}{(}\PY{n}{hyphotesis\PYZus{}function}\PY{p}{,} \PY{n}{X}\PY{p}{,} \PY{n}{Y}\PY{p}{,} \PY{n}{m}\PY{p}{,} \PY{n}{independent\PYZus{}data}\PY{p}{,} \PY{n}{dependent\PYZus{}data}\PY{p}{)}
         
         \PY{n}{ax}\PY{o}{.}\PY{n}{plot\PYZus{}surface}\PY{p}{(}\PY{n}{X}\PY{p}{,} \PY{n}{Y}\PY{p}{,} \PY{n}{Z}\PY{p}{,} \PY{n}{cmap}\PY{o}{=}\PY{n}{cm}\PY{o}{.}\PY{n}{coolwarm}\PY{p}{,} \PY{n}{linewidth}\PY{o}{=}\PY{l+m+mi}{0}\PY{p}{,} \PY{n}{antialiased}\PY{o}{=}\PY{k+kc}{False}\PY{p}{)}
         \PY{n}{ax}\PY{o}{.}\PY{n}{set\PYZus{}xlim}\PY{p}{(}\PY{o}{\PYZhy{}}\PY{l+m+mi}{10}\PY{p}{,} \PY{l+m+mi}{10}\PY{p}{)}
         \PY{n}{ax}\PY{o}{.}\PY{n}{set\PYZus{}ylim}\PY{p}{(}\PY{o}{\PYZhy{}}\PY{l+m+mi}{1}\PY{p}{,} \PY{l+m+mi}{4}\PY{p}{)}
         \PY{n}{plt}\PY{o}{.}\PY{n}{xlabel}\PY{p}{(}\PY{l+s+s1}{\PYZsq{}}\PY{l+s+s1}{Th0}\PY{l+s+s1}{\PYZsq{}}\PY{p}{)}
         \PY{n}{plt}\PY{o}{.}\PY{n}{ylabel}\PY{p}{(}\PY{l+s+s1}{\PYZsq{}}\PY{l+s+s1}{Th1}\PY{l+s+s1}{\PYZsq{}}\PY{p}{)}
\end{Verbatim}


\begin{Verbatim}[commandchars=\\\{\}]
{\color{outcolor}Out[{\color{outcolor}20}]:} Text(0.5,0,'Th1')
\end{Verbatim}
            
    \begin{center}
    \adjustimage{max size={0.9\linewidth}{0.9\paperheight}}{output_38_1.png}
    \end{center}
    { \hspace*{\fill} \\}
    
    \begin{Verbatim}[commandchars=\\\{\}]
{\color{incolor}In [{\color{incolor}22}]:} \PY{n}{plt}\PY{o}{.}\PY{n}{contour}\PY{p}{(}\PY{n}{X}\PY{p}{,} \PY{n}{Y}\PY{p}{,} \PY{n}{Z} \PY{p}{,}\PY{n}{np}\PY{o}{.}\PY{n}{logspace}\PY{p}{(}\PY{o}{\PYZhy{}}\PY{l+m+mi}{2}\PY{p}{,} \PY{l+m+mi}{3}\PY{p}{,} \PY{l+m+mi}{20}\PY{p}{)}\PY{p}{,} \PY{n}{colors}\PY{o}{=}\PY{l+s+s1}{\PYZsq{}}\PY{l+s+s1}{black}\PY{l+s+s1}{\PYZsq{}}\PY{p}{)}
         \PY{n}{plt}\PY{o}{.}\PY{n}{plot}\PY{p}{(}\PY{n}{vc\PYZus{}gd\PYZus{}th0}\PY{p}{[}\PY{p}{:}\PY{p}{:}\PY{l+m+mi}{150}\PY{p}{]}\PY{p}{,} \PY{n}{vc\PYZus{}gd\PYZus{}th1}\PY{p}{[}\PY{p}{:}\PY{p}{:}\PY{l+m+mi}{150}\PY{p}{]}\PY{p}{,} \PY{l+s+s1}{\PYZsq{}}\PY{l+s+s1}{rx}\PY{l+s+s1}{\PYZsq{}}\PY{p}{)}
\end{Verbatim}


\begin{Verbatim}[commandchars=\\\{\}]
{\color{outcolor}Out[{\color{outcolor}22}]:} [<matplotlib.lines.Line2D at 0x7fe6760d84a8>]
\end{Verbatim}
            
    \begin{center}
    \adjustimage{max size={0.9\linewidth}{0.9\paperheight}}{output_39_1.png}
    \end{center}
    { \hspace*{\fill} \\}
    
    \section{Parte 2 : Regresión lineal con dos
variables}\label{parte-2-regresiuxf3n-lineal-con-dos-variables}

    \subsection{Carga y normalización
datos}\label{carga-y-normalizaciuxf3n-datos}

    En esta sección, realizaremos regresión lineal con múltiples variables
utilizando un dataset diferente al utilizado en la sección anterior. Por
ello, realizamos una nueva carga.

    \begin{Verbatim}[commandchars=\\\{\}]
{\color{incolor}In [{\color{incolor}23}]:} \PY{n}{dataset} \PY{o}{=} \PY{n}{load\PYZus{}csv}\PY{p}{(}\PY{l+s+s1}{\PYZsq{}}\PY{l+s+s1}{./datasets/ex1data2.csv}\PY{l+s+s1}{\PYZsq{}}\PY{p}{)}
         \PY{n}{nfeatures} \PY{o}{=} \PY{l+m+mi}{2}
         \PY{n}{dataset}\PY{p}{[}\PY{p}{:}\PY{l+m+mi}{5}\PY{p}{]}
\end{Verbatim}


\begin{Verbatim}[commandchars=\\\{\}]
{\color{outcolor}Out[{\color{outcolor}23}]:} array([[2.104e+03, 3.000e+00, 3.999e+05],
                [1.600e+03, 3.000e+00, 3.299e+05],
                [2.400e+03, 3.000e+00, 3.690e+05],
                [1.416e+03, 2.000e+00, 2.320e+05],
                [3.000e+03, 4.000e+00, 5.399e+05]])
\end{Verbatim}
            
    \begin{Verbatim}[commandchars=\\\{\}]
{\color{incolor}In [{\color{incolor}24}]:} \PY{n}{features} \PY{o}{=} \PY{n}{dataset}\PY{p}{[}\PY{p}{:}\PY{p}{,}\PY{p}{:}\PY{n}{nfeatures}\PY{p}{]} \PY{c+c1}{\PYZsh{} (47,2)}
         \PY{n}{target} \PY{o}{=} \PY{n}{dataset}\PY{p}{[}\PY{p}{:}\PY{p}{,}\PY{n}{nfeatures}\PY{p}{]} \PY{c+c1}{\PYZsh{} (47,)}
         \PY{n}{features}\PY{o}{.}\PY{n}{shape}
\end{Verbatim}


\begin{Verbatim}[commandchars=\\\{\}]
{\color{outcolor}Out[{\color{outcolor}24}]:} (47, 2)
\end{Verbatim}
            
    Normalizamos los datos los datos de las variables características ya que
las unidades utilizadas en el dataset utilizado varían en función al
atributo en cuestión. Esto podría generar problemas por lo que
sustituimos cada valor de X por su división entre su diferencia con la
media de su columna y la desviación estándar de su columna, teniendo en
cuenta que cada columna se corresponde con una variable característica
en cuestión.

Esto queda expresado por:

\[ x_i \leftarrow \frac{x_i - \mu}{\sigma_i} \]

    \begin{Verbatim}[commandchars=\\\{\}]
{\color{incolor}In [{\color{incolor}25}]:} \PY{k}{def} \PY{n+nf}{normalice}\PY{p}{(}\PY{n}{x}\PY{p}{)}\PY{p}{:}
             \PY{n}{mu} \PY{o}{=} \PY{p}{[}\PY{n}{np}\PY{o}{.}\PY{n}{mean}\PY{p}{(}\PY{n}{x}\PY{p}{[}\PY{p}{:}\PY{p}{,}\PY{n}{i}\PY{p}{]}\PY{p}{)} \PY{k}{for} \PY{n}{i} \PY{o+ow}{in} \PY{n+nb}{range}\PY{p}{(}\PY{l+m+mi}{0}\PY{p}{,}\PY{n+nb}{len}\PY{p}{(}\PY{n}{x}\PY{p}{[}\PY{l+m+mi}{0}\PY{p}{]}\PY{p}{)}\PY{p}{)}\PY{p}{]}
             \PY{n}{sigma} \PY{o}{=} \PY{p}{[}\PY{n}{np}\PY{o}{.}\PY{n}{std}\PY{p}{(}\PY{n}{x}\PY{p}{[}\PY{p}{:}\PY{p}{,}\PY{n}{i}\PY{p}{]}\PY{p}{)} \PY{k}{for} \PY{n}{i} \PY{o+ow}{in} \PY{n+nb}{range}\PY{p}{(}\PY{l+m+mi}{0}\PY{p}{,}\PY{n+nb}{len}\PY{p}{(}\PY{n}{x}\PY{p}{[}\PY{l+m+mi}{0}\PY{p}{]}\PY{p}{)}\PY{p}{)}\PY{p}{]}
             \PY{n}{xnorm} \PY{o}{=} \PY{p}{(}\PY{n}{x} \PY{o}{\PYZhy{}} \PY{n}{mu}\PY{p}{)}\PY{o}{/}\PY{n}{sigma}
             \PY{k}{return} \PY{n}{xnorm}\PY{p}{,} \PY{n}{mu}\PY{p}{,} \PY{n}{sigma}
\end{Verbatim}


    \begin{Verbatim}[commandchars=\\\{\}]
{\color{incolor}In [{\color{incolor}26}]:} \PY{n}{features\PYZus{}norm}\PY{p}{,} \PY{n}{mu}\PY{p}{,} \PY{n}{sigma} \PY{o}{=} \PY{n}{normalice}\PY{p}{(}\PY{n}{features}\PY{p}{)}
         \PY{n}{ones} \PY{o}{=} \PY{n}{np}\PY{o}{.}\PY{n}{ones}\PY{p}{(}\PY{p}{[}\PY{n+nb}{len}\PY{p}{(}\PY{n}{features}\PY{p}{)}\PY{p}{,}\PY{n+nb}{len}\PY{p}{(}\PY{n}{features}\PY{p}{[}\PY{l+m+mi}{0}\PY{p}{]}\PY{p}{)}\PY{o}{+}\PY{l+m+mi}{1}\PY{p}{]}\PY{p}{)}
         \PY{n}{ones}\PY{p}{[}\PY{p}{:}\PY{p}{,}\PY{l+m+mi}{1}\PY{p}{:}\PY{p}{]} \PY{o}{=} \PY{n}{features\PYZus{}norm}
         \PY{n}{features\PYZus{}norm} \PY{o}{=} \PY{n}{ones}
         \PY{n}{features\PYZus{}norm}\PY{p}{[}\PY{p}{:}\PY{l+m+mi}{3}\PY{p}{]}
\end{Verbatim}


\begin{Verbatim}[commandchars=\\\{\}]
{\color{outcolor}Out[{\color{outcolor}26}]:} array([[ 1.        ,  0.13141542, -0.22609337],
                [ 1.        , -0.5096407 , -0.22609337],
                [ 1.        ,  0.5079087 , -0.22609337]])
\end{Verbatim}
            
    \begin{Verbatim}[commandchars=\\\{\}]
{\color{incolor}In [{\color{incolor}27}]:} \PY{n}{th} \PY{o}{=} \PY{p}{[}\PY{l+m+mi}{1}\PY{p}{,}\PY{l+m+mi}{1}\PY{p}{,}\PY{l+m+mi}{1}\PY{p}{]}
         \PY{n}{lr} \PY{o}{=} \PY{l+m+mf}{0.01}
         \PY{n}{m} \PY{o}{=} \PY{n+nb}{len}\PY{p}{(}\PY{n}{dataset}\PY{p}{)}
\end{Verbatim}


    \subsection{Función hipótesis}\label{funciuxf3n-hipuxf3tesis}

    Debido a que tenemos más de una variable característica, en este caso
utilizaremos una función definida de la siguiente forma:

\[ h_\theta(x)= \theta^T x \]

    \begin{Verbatim}[commandchars=\\\{\}]
{\color{incolor}In [{\color{incolor}28}]:} \PY{n}{hyphotesis\PYZus{}function} \PY{o}{=} \PY{k}{lambda} \PY{n}{th}\PY{p}{,}\PY{n}{x} \PY{p}{:} \PY{n}{np}\PY{o}{.}\PY{n}{dot}\PY{p}{(}\PY{n}{x}\PY{p}{,} \PY{n}{th}\PY{p}{)}
\end{Verbatim}


    \begin{Verbatim}[commandchars=\\\{\}]
{\color{incolor}In [{\color{incolor}29}]:} \PY{n}{hyphotesis\PYZus{}function}\PY{p}{(}\PY{n}{th}\PY{p}{,} \PY{n}{features\PYZus{}norm}\PY{p}{[}\PY{l+m+mi}{0}\PY{p}{,}\PY{p}{:}\PY{p}{]}\PY{p}{)}
\end{Verbatim}


\begin{Verbatim}[commandchars=\\\{\}]
{\color{outcolor}Out[{\color{outcolor}29}]:} 0.9053220544433592
\end{Verbatim}
            
    \subsection{Función coste}\label{funciuxf3n-coste}

    La función de coste (debajo) queda definida por la siguiente expresión:

\[ J(\theta) = \frac{1}{2m}(X\theta - \vec{y})^{T}(X\theta-\vec{y}) \]

    \begin{Verbatim}[commandchars=\\\{\}]
{\color{incolor}In [{\color{incolor}30}]:} \PY{k}{def} \PY{n+nf}{cost\PYZus{}function}\PY{p}{(}\PY{n}{fun}\PY{p}{,} \PY{n}{th}\PY{p}{,} \PY{n}{m}\PY{p}{,} \PY{n}{x}\PY{p}{,} \PY{n}{y}\PY{p}{)}\PY{p}{:}
             \PY{n}{cost} \PY{o}{=} \PY{p}{(}\PY{n}{x}\PY{o}{.}\PY{n}{dot}\PY{p}{(}\PY{n}{th}\PY{p}{)} \PY{o}{\PYZhy{}} \PY{n}{y}\PY{p}{)}\PY{o}{.}\PY{n}{T}\PY{o}{.}\PY{n}{dot}\PY{p}{(}\PY{n}{x}\PY{o}{.}\PY{n}{dot}\PY{p}{(}\PY{n}{th}\PY{p}{)} \PY{o}{\PYZhy{}} \PY{n}{y}\PY{p}{)} \PY{o}{/} \PY{p}{(}\PY{l+m+mi}{2}\PY{o}{*}\PY{n}{m}\PY{p}{)}
             \PY{k}{return} \PY{n}{cost}
\end{Verbatim}


    \begin{Verbatim}[commandchars=\\\{\}]
{\color{incolor}In [{\color{incolor}31}]:} \PY{n}{cost\PYZus{}function}\PY{p}{(}\PY{n}{hyphotesis\PYZus{}function}\PY{p}{,} \PY{n}{th}\PY{p}{,} \PY{n}{m}\PY{p}{,} \PY{n}{features\PYZus{}norm}\PY{p}{,} \PY{n}{target}\PY{p}{)}
\end{Verbatim}


\begin{Verbatim}[commandchars=\\\{\}]
{\color{outcolor}Out[{\color{outcolor}31}]:} 65591047222.902596
\end{Verbatim}
            
    \subsection{\texorpdfstring{\emph{Gradient
descent}}{Gradient descent}}\label{gradient-descent}

    La función \emph{gradient descent} (debajo) actualiza los valores de
\(\theta_j\) \emph{epochs} veces siguiendo la siguiente expresión:

\[ \theta_j := \theta_j - \alpha\frac{1}{m}\sum_{i=1}^{m}(h_{\theta}(x^{(i)}) - y^{(i)}){x_{j}^{(i)}} \]

    \begin{Verbatim}[commandchars=\\\{\}]
{\color{incolor}In [{\color{incolor}32}]:} \PY{k}{def} \PY{n+nf}{gradient\PYZus{}descent}\PY{p}{(}\PY{n}{fun}\PY{p}{,} \PY{n}{th}\PY{p}{,} \PY{n}{m}\PY{p}{,} \PY{n}{x}\PY{p}{,} \PY{n}{y}\PY{p}{,} \PY{n}{lr}\PY{o}{=}\PY{l+m+mf}{0.01}\PY{p}{,} \PY{n}{epochs}\PY{o}{=}\PY{l+m+mi}{1500}\PY{p}{)}\PY{p}{:}
             \PY{n}{cost} \PY{o}{=} \PY{p}{[}\PY{p}{]}
             \PY{n}{curr\PYZus{}th} \PY{o}{=} \PY{n}{th}
             \PY{k}{for} \PY{n}{e} \PY{o+ow}{in} \PY{n+nb}{range}\PY{p}{(}\PY{l+m+mi}{0}\PY{p}{,} \PY{n}{epochs}\PY{p}{)}\PY{p}{:}        
                 \PY{n}{curr\PYZus{}th} \PY{o}{=} \PY{n}{curr\PYZus{}th} \PY{o}{\PYZhy{}} \PY{p}{(}\PY{l+m+mi}{1}\PY{o}{/}\PY{n}{m}\PY{p}{)}\PY{o}{*}\PY{n}{lr}\PY{o}{*}\PY{p}{(}\PY{n}{np}\PY{o}{.}\PY{n}{dot}\PY{p}{(}\PY{n}{x}\PY{o}{.}\PY{n}{T}\PY{p}{,} \PY{n}{fun}\PY{p}{(}\PY{n}{curr\PYZus{}th}\PY{p}{,} \PY{n}{x}\PY{p}{)} \PY{o}{\PYZhy{}} \PY{n}{y}\PY{p}{)}\PY{p}{)}
                 \PY{n}{epoch\PYZus{}cost} \PY{o}{=} \PY{n}{cost\PYZus{}function}\PY{p}{(}\PY{n}{fun}\PY{p}{,} \PY{n}{curr\PYZus{}th}\PY{p}{,} \PY{n}{m}\PY{p}{,} \PY{n}{x}\PY{p}{,} \PY{n}{y}\PY{p}{)}
                 \PY{n}{cost} \PY{o}{+}\PY{o}{=} \PY{p}{[}\PY{n}{epoch\PYZus{}cost}\PY{p}{]}
                 \PY{c+c1}{\PYZsh{}print(\PYZsq{}It: \PYZob{}\PYZcb{}, Cost: \PYZob{}\PYZcb{}\PYZsq{}.format(e + 1, epoch\PYZus{}cost))}
             \PY{k}{return} \PY{n}{curr\PYZus{}th}\PY{p}{,} \PY{n}{cost}   
\end{Verbatim}


    \begin{Verbatim}[commandchars=\\\{\}]
{\color{incolor}In [{\color{incolor}33}]:} \PY{n}{gd\PYZus{}th}\PY{p}{,} \PY{n}{gd\PYZus{}cost} \PY{o}{=} \PY{n}{gradient\PYZus{}descent}\PY{p}{(}\PY{n}{hyphotesis\PYZus{}function}\PY{p}{,} \PY{n}{th}\PY{p}{,}
                                           \PY{n}{m}\PY{p}{,} \PY{n}{features\PYZus{}norm}\PY{p}{,} \PY{n}{target}\PY{p}{,} \PY{n}{lr}\PY{o}{=}\PY{l+m+mf}{0.01}\PY{p}{,} \PY{n}{epochs}\PY{o}{=}\PY{l+m+mi}{1500}\PY{p}{)}
         \PY{n}{gd\PYZus{}th}
\end{Verbatim}


\begin{Verbatim}[commandchars=\\\{\}]
{\color{outcolor}Out[{\color{outcolor}33}]:} array([340412.56301468, 109370.05670466,  -6500.61509507])
\end{Verbatim}
            
    Mostramos la grafica donde se ve la disminucion de costes:

    \begin{Verbatim}[commandchars=\\\{\}]
{\color{incolor}In [{\color{incolor}34}]:} \PY{n}{plt}\PY{o}{.}\PY{n}{figure}\PY{p}{(}\PY{p}{)}
         \PY{n}{plt}\PY{o}{.}\PY{n}{plot}\PY{p}{(}\PY{n}{gd\PYZus{}cost}\PY{p}{)}
         \PY{n}{plt}\PY{o}{.}\PY{n}{title}\PY{p}{(}\PY{l+s+s1}{\PYZsq{}}\PY{l+s+s1}{Coste por iteración gradient\PYZus{}descent()}\PY{l+s+s1}{\PYZsq{}}\PY{p}{)}
         \PY{n}{plt}\PY{o}{.}\PY{n}{ylabel}\PY{p}{(}\PY{l+s+s1}{\PYZsq{}}\PY{l+s+s1}{Cost}\PY{l+s+s1}{\PYZsq{}}\PY{p}{)}
         \PY{n}{plt}\PY{o}{.}\PY{n}{xlabel}\PY{p}{(}\PY{l+s+s1}{\PYZsq{}}\PY{l+s+s1}{num\PYZus{}epoch}\PY{l+s+s1}{\PYZsq{}}\PY{p}{)}
\end{Verbatim}


\begin{Verbatim}[commandchars=\\\{\}]
{\color{outcolor}Out[{\color{outcolor}34}]:} Text(0.5,0,'num\_epoch')
\end{Verbatim}
            
    \begin{center}
    \adjustimage{max size={0.9\linewidth}{0.9\paperheight}}{output_62_1.png}
    \end{center}
    { \hspace*{\fill} \\}
    
    \subsection{\texorpdfstring{\emph{Normal
Equation}}{Normal Equation}}\label{normal-equation}

    De forma alternativa a \emph{gradient descent}, planteamos la expresión
de la ecuación normal que resuelve los valores de \(\theta\) siguiendo
la siguiente expresión:

\[ \theta = (X^{T}X)^{-1}X^{T}\vec{y} \]

    \begin{Verbatim}[commandchars=\\\{\}]
{\color{incolor}In [{\color{incolor}35}]:} \PY{n}{ones} \PY{o}{=} \PY{n}{np}\PY{o}{.}\PY{n}{ones}\PY{p}{(}\PY{p}{[}\PY{n+nb}{len}\PY{p}{(}\PY{n}{features}\PY{p}{)}\PY{p}{,}\PY{n+nb}{len}\PY{p}{(}\PY{n}{features}\PY{p}{[}\PY{l+m+mi}{0}\PY{p}{]}\PY{p}{)}\PY{o}{+}\PY{l+m+mi}{1}\PY{p}{]}\PY{p}{)}
         \PY{n}{ones}\PY{p}{[}\PY{p}{:}\PY{p}{,}\PY{l+m+mi}{1}\PY{p}{:}\PY{p}{]} \PY{o}{=} \PY{n}{features}
         \PY{n}{features\PYZus{}wones} \PY{o}{=} \PY{n}{ones}
         \PY{n}{features\PYZus{}t} \PY{o}{=} \PY{n}{np}\PY{o}{.}\PY{n}{transpose}\PY{p}{(}\PY{n}{features\PYZus{}wones}\PY{p}{)}
         \PY{n}{th\PYZus{}norm\PYZus{}eq} \PY{o}{=} \PY{n}{np}\PY{o}{.}\PY{n}{dot}\PY{p}{(}\PY{n}{np}\PY{o}{.}\PY{n}{dot}\PY{p}{(}\PY{n}{np}\PY{o}{.}\PY{n}{linalg}\PY{o}{.}\PY{n}{pinv}\PY{p}{(}\PY{n}{features\PYZus{}t}\PY{o}{.}\PY{n}{dot}\PY{p}{(}\PY{n}{features\PYZus{}wones}\PY{p}{)}\PY{p}{)}\PY{p}{,} \PY{n}{features\PYZus{}t}\PY{p}{)}\PY{p}{,} \PY{n}{target}\PY{p}{)}
         \PY{n+nb}{print} \PY{p}{(}\PY{n}{th\PYZus{}norm\PYZus{}eq}\PY{p}{)}
\end{Verbatim}


    \begin{Verbatim}[commandchars=\\\{\}]
[89597.90954361   139.21067402 -8738.01911255]

    \end{Verbatim}

    \subsubsection{Resultados}\label{resultados}

    Para comprobar que el modelo implementado utilizando \emph{gradient
descent} es correcto, realizamos una predicción con los valores
obtenidos de \(\theta\) con nuestra implementación y los valores
obtenidos con la ecuación normal

    \begin{Verbatim}[commandchars=\\\{\}]
{\color{incolor}In [{\color{incolor}37}]:} \PY{n}{x\PYZus{}test} \PY{o}{=} \PY{p}{[}\PY{l+m+mi}{1}\PY{p}{,} \PY{l+m+mi}{1650}\PY{p}{,} \PY{l+m+mi}{3}\PY{p}{]}
         \PY{n}{x\PYZus{}test}\PY{p}{[}\PY{l+m+mi}{1}\PY{p}{:}\PY{p}{]} \PY{o}{=} \PY{p}{(}\PY{n}{np}\PY{o}{.}\PY{n}{array}\PY{p}{(}\PY{n}{x\PYZus{}test}\PY{p}{[}\PY{l+m+mi}{1}\PY{p}{:}\PY{p}{]}\PY{p}{)} \PY{o}{\PYZhy{}} \PY{n}{mu} \PY{p}{)} \PY{o}{/} \PY{n}{sigma}
         \PY{n+nb}{print}\PY{p}{(}\PY{l+s+s1}{\PYZsq{}}\PY{l+s+s1}{Predicción gradient descent:}\PY{l+s+si}{\PYZob{}\PYZcb{}}\PY{l+s+s1}{, predicción ecuación normal:}\PY{l+s+si}{\PYZob{}\PYZcb{}}\PY{l+s+s1}{\PYZsq{}}\PY{o}{.}
               \PY{n+nb}{format}\PY{p}{(}\PY{n}{hyphotesis\PYZus{}function}\PY{p}{(}\PY{n}{gd\PYZus{}th}\PY{p}{,} \PY{n}{x\PYZus{}test}\PY{p}{)}\PY{p}{,}
                      \PY{n}{hyphotesis\PYZus{}function}\PY{p}{(}\PY{n}{th\PYZus{}norm\PYZus{}eq}\PY{p}{,} \PY{p}{[}\PY{l+m+mi}{1}\PY{p}{,} \PY{l+m+mi}{1650}\PY{p}{,} \PY{l+m+mi}{3}\PY{p}{]}\PY{p}{)}\PY{p}{)}\PY{p}{)}
\end{Verbatim}


    \begin{Verbatim}[commandchars=\\\{\}]
Predicción gradient descent:293098.4666760489, predicción ecuación normal:293081.4643349892

    \end{Verbatim}


    % Add a bibliography block to the postdoc
    
    
    
    \end{document}
